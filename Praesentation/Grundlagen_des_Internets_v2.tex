%hierbei handelt es sich um die Präsentation imt .tex-Format
%Abbildungen etc. müssen entsprechend in eigenen Unterordnern präsent sein!

\documentclass[utf8]{beamer}

\mode<presentation>
{
	\usetheme{Warsaw}
	% oder ...
	
	\setbeamercovered{transparent}
	% oder auch nicht
}

\usepackage[ngerman]{babel}
\usepackage[T1]{fontenc}
\usepackage{times}
\usepackage[babel, german=quotes]{csquotes}

\title{Grundlagen des Internets und Sicherheit im Web}
\subtitle{Das Böse im Netz und was man dagegen tun kann} % (optional)
\author{Georg A. Murzik, Marcus Schilling}
\institute{Terminal.21}

\date{05. November 2016}

% Dies wird lediglich in den PDF Informationskatalog einfügt. Kann gut
% weggelassen werden.
\subject{Grundlagen des Internets v2}

%Logo der MArtin-Luther Universität
%\pgfdeclareimage[height=0.5cm]{Logo_Uni}{Abbildungen/Logo_Universitaet}
%\logo{\pgfuseimage{Logo_Uni}}

% Folgendes sollte gelöscht werden, wenn man nicht am Anfang jedes
% Unterabschnitts die Gliederung nochmal sehen möchte.
%\AtBeginSubsection[]
%{
%  \begin{frame}<beamer>{Gliederung}
%    \tableofcontents[currentsection,currentsubsection]
%  \end{frame}
%}

%Foliennummern
\setbeamertemplate{footline}
{%
	\begin{beamercolorbox}[wd=0.5\textwidth,ht=3ex,dp=1.5ex,leftskip=.5em,rightskip=.5em]{author in head/foot}%
		\usebeamerfont{author in head/foot}%
		\insertframenumber{} von \inserttotalframenumber\hfill\insertshortauthor%
	\end{beamercolorbox}%
	\vspace*{-4.5ex}\hspace*{0.5\textwidth}%
	\begin{beamercolorbox}[wd=0.5\textwidth,ht=3ex,dp=1.5ex,left,leftskip=.5em]{title in head/foot}%
		\usebeamerfont{title in head/foot}%
		\insertshorttitle%
	\end{beamercolorbox}%
}

%%%%%%%%%%%%%%%%%%%%%%%%%%%%%%%%%%%%%%%%%%%%%%%%%%%%%%%%%%%%%%%%%%%%%%%%%%%%%%%%%%%%%%%%%%%%%%%
%%%                                Begin der Präsentation                                        %%%
%%%%%%%%%%%%%%%%%%%%%%%%%%%%%%%%%%%%%%%%%%%%%%%%%%%%%%%%%%%%%%%%%%%%%%%%%%%%%%%%%%%%%%%%%%%%%%%

\begin{document}
	
	\begin{frame}
		\titlepage
	\end{frame}
	
	\section{Grundlagen des Internets}
	\subsection{Aufbau des Internets}
	\begin{frame}{Übersicht}
		
	\end{frame}
	
	\begin{frame}{Hosts}
		
	\end{frame}
	
	\begin{frame}{\enquote{Letzte Meile}}
		
	\end{frame}
	
	\begin{frame}{PoP}
		
	\end{frame}
	
	\begin{frame}{ISP}
		
	\end{frame}
	
	\begin{frame}{IXP}
		
	\end{frame}
	
	\begin{frame}{Sonstige Netze}
		
	\end{frame}
	
	\subsection{Kommunikation}
	\begin{frame}{Webdienste}
		
	\end{frame}
	
	\begin{frame}{DNS}
		
	\end{frame}
	
	\begin{frame}{Webclients}
		
	\end{frame}
	
	\begin{frame}{Aufruf einer Webseite}
		
	\end{frame}
	
	\section{Gefahren im Netz}
	\subsection{Datendiebstahl}
	
	\subsection{Phishing}
	
	\subsection{Tracking}
	\subsubsection{Datenschutz}
	\begin{frame}{Hintergrundwissen Datenschutz} 
		\begin{itemize}
			\item “Privacy protects us from abuses by those in power, even if we’re doing nothing wrong at the time of surveillance.” (Bruce Schneier)
			\item Welche Daten werden für welchen Verwendungszweck wo erhoben, wie verarbeitet und an wen weitergegeben?
			\item Tradeoff: Keine Daten anfallen lassen. vs. Dienste bequem in Anspruch nehmen
			\note{ Was sind die offiziellen Spielregeln?}
			\note{ Welche juristischen und technischen Möglichkeiten habe ich?}
		\end{itemize}
	\end{frame}
	
	\begin{frame}{Hintergrundwissen Datenschutz: Spielregeln}
		\begin{itemize}
			\item personenbezogenen Daten: Informationen, die mit einer Person verbunden werden
			\item Verwendung: Sammeln, verarbeiten, weitergeben
			\item Kenntlichmachung von notwendigen und freiwilligen Angaben
			\item Aufklärung über etwaige Nachteile, wenn man freiwillige Angaben nicht tätigt
			\item Einwilligung schriftlich oder digital
			\item neuer Verwendungszweck = neue Erlaubnis
		\end{itemize}
	\end{frame}
	
	\begin{frame}{Hintergrundwissen Datenschutz: Rechte}
		\begin{itemize}
			\item Informationelle Selbstbestimmung
			\item Recht auf Auskunft: Was wird gespeichert, wo kommt es her, wo geht es hin?
			\item Recht auf Berichtigung: Wenn was falsch ist.
			\item Recht auf Löschung: Wenn die Daten für den vereinbarten Zweck nicht mehr notwendig sind.
		\end{itemize}
	\end{frame}
	
	\begin{frame}{Hintergrundwissen Datenschutz: Praxis}
		\begin{itemize}
			\item viele Länder = viele Gesetze
			\item Es ist praktisch umstritten, welche Daten personenbezogen sind (z.B. Gerätekennung vs. natürliche Person)
			\item wir hinterlassen beim Browsen viele Spuren, die nebenbei verarbeitet werden; durch viele Informationsquellen kann man auf die Person schließen
			\note{Welche Techniken des Trackings werden beim genutzt?}
			\note{Welche Maßnahmen kann man zum Schutz meiner Privatsphäre ergreifen?}
		\end{itemize}
	\end{frame}
	
	\begin{frame}{Tracking beim Verbindungsaufbau}
		\begin{itemize}
			\item DNS-Anfragen
			\item IP-Adressen Routing
			\item Social-media-like buttons
			\item Web-beacons 
		\end{itemize}
	\end{frame}
	
	\begin{frame}{Tracking durch verräterischen Browser}
		\begin{itemize}
			\item Geodaten
			\item Browserfingerprinting
			\item Cookies in verschiedenen Geschmacksrichtungen
			\item URLs 
		\end{itemize}
	\end{frame}
	
	\begin{frame}
		\begin{itemize}
			\item 
		\end{itemize}
	\end{frame}
	
	\begin{frame}
		\begin{itemize}
			\item 
		\end{itemize}
	\end{frame}
	
	\subsubsection{\enquote{Angewandter Datenschutz} im Internet}
	
	\section{Counterstrike GO}
	\subsection{Verhalten im Netz}
	\begin{frame}{Verhalten im Netz}
		\begin{itemize}
			\item Datensparsamkeit
			\begin{itemize}
				\item Schein vs. Sein
				\item Nicht überall registrieren, nicht eingeloggt bleiben
			\end{itemize}
			\item Kostenlose Dienste hinterfragen
			\begin{itemize}
				\item Suchmaschinen:
				Jede Suche ist mit einer Motivation verbunden, die Rückschlüssel zulässt. https://duckduckgo.com ist gut.
				\item Email:
				[web / gmx / aol / google / yahoo / hotmail ...] Sind doof, weil die Nachrichten mit kommerziellen Interessen durchsucht werden. Besser: [mailbox.org / poesto.de]
				\item Cloud:
				Was ich nicht kontrollieren kann ist schon verloren. Würdet ihr wirklich gerne euren privaten Urlaubsfotos auf anderen Computern speichern, auf die mindestens irgendwelche Mitarbeit zugriff haben? Und außerdem: Irgendwann werden alle gehackt. Wenn, dann die Daten sehr ordentlich verschlüsseln.
			\end{itemize}
		\end{itemize}
	\end{frame}
	
	\subsection{Technische Hilfsmittel}
	\begin{frame}{Addons}
		\noindent
		\begin{description}
			\item[Privaconf]{https://addons.mozilla.org/de/firefox/addon/privaconf/}
			\item[Cookie-Controller]{https://addons.mozilla.org/de/firefox/addon/cookie-controller/}
			\item[Canvasblocker]{https://addons.mozilla.org/de/firefox/addon/canvasblocker/}
			\item[no-rescue-uri-leak]{https://addons.mozilla.org/de/firefox/addon/no-resource-uri-leak/}
			\item[UserAgentChanger]{Nicht benutzen!}
		\end{description}
	\end{frame}
	
	% Da dies ein Vorlage für beliebige Vorträge ist, lassen sich kaum
	% allgemeine Regeln zur Strukturierung angeben. Da die Vorlage für
	% einen Vortrag zwischen 15 und 45 Minuten gedacht ist, fährt man aber
	% mit folgenden Regeln oft gut.  
	
	% - Es sollte genau zwei oder drei Abschnitte geben (neben der
	%   Zusammenfassung). 
	% - *Höchstens* drei Unterabschnitte pro Abschnitt.
	% - Pro Rahmen sollte man zwischen 30s und 2min reden. Es sollte also
	%   15 bis 30 Rahmen geben.
	
	
\end{document}


