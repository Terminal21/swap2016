% Todo:
% - ganz am Anfang ggf. Logos einbetten: https://download.terminal21.de/terminal21/logo/logo_aktuell_schwarz.svg
% - anschauliche Linkliste
% - ganz am Ende wäre ggf. ein Quellen-Nachweis aller Bilder schön

%hierbei handelt es sich um die Präsentation imt .tex-Format
%Abbildungen etc. müssen entsprechend in eigenen Unterordnern präsent sein!

\documentclass[hyperref={colorlinks,linkcolor=blue}, utf8]{beamer}

\mode<presentation>
{
	\usetheme{Warsaw}
	% oder ...
	
	\setbeamercovered{transparent}
	% oder auch nicht
}

\usepackage[ngerman]{babel}
\usepackage[utf8]{inputenc}
\usepackage[T1]{fontenc}
\usepackage{times}
\usepackage[babel, german=quotes]{csquotes}


\title{Grundlagen des Internets und Sicherheit im Web}
\subtitle{Das Böse im Netz und was man dagegen tun kann} % (optional)
\author{Georg A. Murzik, Marcus Schilling}
\institute{Terminal.21}

\date{05. November 2016}

% Dies wird lediglich in den PDF Informationskatalog einfügt. Kann gut
% weggelassen werden.
\subject{Grundlagen des Internets v2}

%Logo vom Terminal.21
\pgfdeclareimage[height=1.0cm]{Logo_Terminal}{Abbildungen/Logo_Terminal}
\logo{\pgfuseimage{Logo_Terminal}}

% Folgendes sollte gelöscht werden, wenn man nicht am Anfang jedes
% Unterabschnitts die Gliederung nochmal sehen möchte.
%\AtBeginSubsection[]
%{
%  \begin{frame}<beamer>{Gliederung}
%    \tableofcontents[currentsection,currentsubsection]
%  \end{frame}
%}

%Foliennummern
\setbeamertemplate{footline}
{%
	\begin{beamercolorbox}[wd=0.5\textwidth,ht=3ex,dp=1.5ex,leftskip=.5em,rightskip=.5em]{author in head/foot}%
		\usebeamerfont{author in head/foot}%
		\insertframenumber{} von \inserttotalframenumber\hfill\insertshortauthor%
	\end{beamercolorbox}%
	\vspace*{-4.5ex}\hspace*{0.5\textwidth}%
	\begin{beamercolorbox}[wd=0.5\textwidth,ht=3ex,dp=1.5ex,left,leftskip=.5em]{title in head/foot}%
		\usebeamerfont{title in head/foot}%
		\insertshorttitle%
	\end{beamercolorbox}%
}

%%%%%%%%%%%%%%%%%%%%%%%%%%%%%%%%%%%%%%%%%%%%%%%%%%%%%%%%%%%%%%%%%%%%%%%%%%%%%%%%%%%%%%%%%%%%%%%
%%%                                Begin der Präsentation                                        %%%
%%%%%%%%%%%%%%%%%%%%%%%%%%%%%%%%%%%%%%%%%%%%%%%%%%%%%%%%%%%%%%%%%%%%%%%%%%%%%%%%%%%%%%%%%%%%%%%

\begin{document}
	
	\begin{frame}
		\titlepage
	\end{frame}
	
	\section{Grundlagen des Internets}
	\begin{frame}
		\centering \huge \textbf{Grundlagen des Internets}
	\end{frame}
	
	\subsection{Aufbau des Internets}
	\begin{frame}{Das wichtigste zuerst}
		\begin{itemize}
			\item Internet ist keine Wolke!
			\item Alles wird von Computern gesteuert.
			\item Jeder Computer gehört jemandem.
			\item Jeder verfolgt eigene Interessen.
			\item Es sind stets wesentlich mehr Personen beteiligt, als es scheint.				\item Daten, die sichtbar sind, werden gelesen.	
		\end{itemize}		
	\end{frame}
	
	\begin{frame}{Woraus besteht das Web?}
		\begin{enumerate}
			\item Webdienste
			\item DNS
			\item Provider
			\item Clients / Hosts
		\end{enumerate}
	\end{frame}
		
	\begin{frame}{Webdienste}
		\begin{itemize}
			\item Bekannteste Dienste sind Webseiten
			\item Cloudspeicher, Handyapps, Navigation, Streaminganbieter sind ebenfalls Webdienste
			\item Webdienste sind über IP-Adressen zu erreichen
			\item IP-Adressen kann sich aber keiner merken, deswegen werden normalerweise URLs verwendet.
		\end{itemize}		
	\end{frame}
	
	\begin{frame}{Webclients}
		\begin{itemize}
			\item Schnittstelle zwischen Nutzer und Internet
			\item Bekannteste Clients sind Webbrowser
			\item Clients stellen Daten von Webservern übersichtlich dar und bieten Interaktionsmöglichkeiten mit dem Server
		\end{itemize}
	\end{frame}
	
	\begin{frame}{DNS}
		\begin{itemize}
			\item DNS (Domain Name Service) dient als eine Art Telefonbuch des Internets
			\item Wandelt Domains in URLs in IP-Adressen um
			\item Jeder Computer, der mit dem Internet kommunizieren möchte, muss mindestens einen DNS kennen.
		\end{itemize}
	
		\begin{figure}[H]
			\includegraphics[width=0.35\textwidth]{Abbildungen/Browser_DNS_Server_Communication}
		\end{figure}				
	\end{frame}
	
	\begin{frame}{Wer hängt zwischen mir und der Webseite?}
		\begin{figure}[H]
			\includegraphics[width=\textwidth]{Abbildungen/Uebersicht_Internet}
			\label{fig:Übersicht des Internets}
		\end{figure}
	\end{frame}
	
	\section{Gefahren im Netz}
	\begin{frame}
		\centering \huge \textbf{Gefahren im Netz}
	\end{frame}
	
	\begin{frame}
		Das Internet wurde ursprünglich unter der Prämisse gebaut, dass nur wenige Menschen in der Lage sein werden, es zu nutzen. Die ursprüngliche Zielgruppe bestand hauptsächlich aus Forschern, die ihre Studien mit anderen Universitäten oder auch innerhalb einer Universität austauschen wollen.
		
		Es wurde daher kaum auf eine sichere Infrastruktur wert gelegt und die Protokolle von heute entsprechen immer noch denen von damals. Sicherheitsstrukturen wurden erst nachträglich eingebaut und erinnern eher an Flicken.
	\end{frame}

	\begin{frame}
		Zusätzlich zu der löchrigen Infrastruktur des Internets werden auch viele Webdienste im Internet stümperhaft programmiert. Diese Kombination führt zu allerlei Gefahren im Internet, denen man sich als normaler Nutzer oftmals nicht bewusst ist.
		
		Genau um diese Gefahren soll es nun gehen.
	\end{frame}

	\begin{frame}{Übersicht der Gefahren im Netz}
		Unabhängig von der Art der Schwachstelle ist die Hauptgefahr immer die selbe: Dass Daten in die falschen Hände geraten. Solche Daten könnten z.B. Bank-, Login-, oder Adressdaten sowie allgemeine persönliche Daten sein. 
		
		Es gibt zudem viele Wege, wie man als Angreifer an solche Informationen gelangen kann:
		\begin{enumerate}
			\item Phishing
			\item Ausnutzen von Schwachstellen
			\item Tracking			
		\end{enumerate}
	\end{frame}

	\subsection{Phishing}
	\begin{frame}
		Als Phishing bezeichnet man den Versuch, an persönliche Daten eines Opfers zu gelangen, indem man ihn auf gefälschte Webseiten leitet, ihm falsche E-Mails oder Kurznachrichten sendet.
		Opfer eines solchen Angriffes ist wahrscheinlich schon jeder einmal gewesen, der im Web unterwegs war. Angriffe reichen von sehr plumpen Versuchen wie \enquote{Sie sind der 1.000.000. Benutzer, holen sie sich hier ihren Gewinn ab.} bis zu aufwändig gestalteten E-Mails, dass man mal eben online bei seiner Bank eine verdächtige Aktivität überprüfen soll, damit kein Geld verloren geht.
	\end{frame}

	\begin{frame}
		Wie einfach es ist, einen solchen Angriff zu starten, möchte ich nun vorführen.
	\end{frame}
	
	\begin{frame}{Abwehr von Phishing-Attacken}
		Die eben erstellte Webseite unterscheidet sich von der originalen Webseite in folgenden Eigenschaften:		
		\begin{description}
			\item[URL] {In der Live-Demo hatte ich keine Zeit, eine richtige Domain für meine Webseite zu registrieren. Hätte ich diese gehabt, so hätte ich eine genommen, die der originalen zum Verwechseln ähnlich sieht, z.B. gooogle.de statt google.de.
			Die Domain ist teil der URL, die oben im Browser steht.
			Hier wäre es einfach zu durchschauen, dass die Seite nicht echt ist - schließlich steht hier nur eine einfache IP-Adresse.}
		\end{description}
	\end{frame}

	\begin{frame}{Abwehr von Phishing-Attacken II}
		\begin{description}
			\item[TLS-Verschlüsselung] {In der Live-Demo hatte ich ebenfalls keine Zeit, mir ein Zertifikat für meinen Server zu besorgen. Dieses sorgt dafür, dass die Verbindung mittels TLS verschlüsselt und über die Zertifizierungsstelle authorisiert ist. Kurz gesagt, dass sichergestellt ist, dass die Webseite die ist, für die sie sich ausgibt und die Verbindung stets verschlüsselt und nicht durch andere manipulierbar ist. Dies erkennt man an dem Schlosssymbol und der grün hinterlegten URL des Browsers.}			
		\end{description}
	\end{frame}
	
	\begin{frame}{Abwehr von Phishing-Attacken III}
		Möchte man sichergehen, dass das Zertifikat einer Webseite vertrauenswürdig ist, so muss man sich die URL und das Zertifikat selbst genau ansehen. Ein Klick auf das Schlosssymbol sollte dies in jedem Browser ermöglichen.
		Im Zertifikat muss der korrekte Name der Firma stehen, die die Seite betreibt und die URL muss genau die sein, die man erwartet. Fast jede große und bekannte Webseite stellt ihre Identitätsdaten im Zertifikat zur Verfügung.
	\end{frame}
	
	\begin{frame}{Abwehr von Phishing-Attacken IV}
		\begin{description}
			\item[Verhalten] {Die Webseite verhält sich anders, als die normale. Gibt man beispielsweise seine Zugangsdaten ein, so wird man nicht eingeloggt, sondern landet wieder auf der (diesmal originalen) Login-Seite des Anbieters. Wenn man das bemerkt, ist es allerdings schon zu spät.}
		\end{description}
	\end{frame}
	
	\begin{frame}{Zusammenfassung}
		Grundsätzlich können Phishing-Attacken durch etwas Misstrauen im Netz leicht abgewehrt werden. Erwarte ich eine Mail von diesem Anbieter? Würde mir ein Anwalt so etwas wirklich per Mail schicken? Wäre wirklich jemand so großzügig, mir das neueste iPhone einfach so zu schenken?
		
		Ist dieser Test bestanden, so gibt es eine weitere Möglichkeit, Phishing zu verhindern: Bekomme ich eine sehr echt aussehende Mail von meiner Bank, so drücke ich auf gar keinen Fall irgendeinen Link darin. Stattdessen mache ich den Browser selbst auf und logge mich in der mir bekannten Webseite der Bank selbst ein. Idealerweise habe ich für solch wichtige Webseiten sowieso ein Lesezeichen gesetzt.
	\end{frame}
	
	\subsection{Ausnutzen von Schwachstellen}
	\begin{frame}
		Deutlich schwieriger abzuwehren sind solche Angriffe, die \emph{Schwachstellen} der Internetseite selbst, des Browsers oder sonstige Programme des Computers ausnutzen.
	\end{frame}

	\begin{frame}{Was sind \enquote{Schwachstellen}!?}
		Schwachstellen sind nichts anderes als Programmierfehler, genauer gesagt, Anwendungsfälle, die der oder die Programmierer nicht bedacht haben. 
		
		Bekommen Programme andere Eingaben, als erwartet und können sie damit nicht korrekt umgehen, so stürzen sie meist ab oder versuchen, die Daten so einzulesen, wie sie es eigentlich nicht tun sollten. Handelt es sich bei einem erwarteten Nutzernamen auf einer Webseite plötzlich um Programmcode, so sollte der Server der Webseite diesen natürlich nicht ausführen. Viele machen das aber (Stichwort: \emph{SQL Injection}).		
	\end{frame}

	\begin{frame}{Was kann passieren?}
		Werden solche Lücken ausgenutzt, so bedeutet das, dass spezieller Code bspw. in präparierten Webseiten direkt durch den Browser oder gar das Betriebssystem selbst ausgeführt wird und bspw. einen Trojaner installiert. Gelangt dieser auf den Rechner des Nutzers, entstehen erheblich größere Gefahren (Ausspähen der Nutzeraktivität und Dateien auf dem Computer usw.).
	\end{frame}

	\begin{frame}{Einschub: Trojaner vs. Viren}
		Viren sind darauf angewiesen, vom Nutzer selbst heruntergeladen zu werden. Sie sind versteckt in vertrauenserweckenden Dateien verschiedener Tauschbörsen wie Videos, Bilder oder besonders kostengünstiger Spiele alternativer Downloadplattformen.
		
		Sie können aber auch in Anhängen von Mails wie PDF- oder MS Office-Dateien lauern und sich dann beim Öffnen nebenher installieren, wenn sie denn eine Schwachstelle in Office oder dem PDF-Viewer ausnutzen können.
	\end{frame}

	\begin{frame}{Einschub: Trojaner vs. Viren II}
		Trojaner sind bösartige Programme, die sich in anderen Programmen verstecken und böses auf dem Computer anrichten. Sie können sich mit dem Internet verbinden, mit ihrem Erschaffer kommunizieren und Befehle ausführen oder einfach fleißig Daten vom Rechner klauen.
		
		Sie sind gerne versteckt in diversen Crackprogrammen und Keygeneratoren, nach denen viele sogar gezielt suchen.
	\end{frame}

	\begin{frame}{Schwachstellen im Browser}
		Browser setzen vorrangig auf Kommunikation mit dem Internet, das Öffnen von Dateien aus dem Internet und dem Ausführen kleiner Programme (Javascript) aus dem Internet.
		
		Klingt nach der vollen Drönung.
		
		Gerade Browser besitzen daher viele Angriffsflächen für Schadcode. 
		\begin{description}
			\item[AddOns] {Addons sind Erweiterungen des Browsers, die Zugriff auf die Inhalte der Webseite bekommen, bevor der Nutzer sie zu sehen bekommt. Ihre Fähigkeiten reichen von der Integration zusätzlicher Emojis über das Blockieren von Werbung bis hin zum Umleiten der gesamten Kommunikation über Proxies. 
			
			Präparierte Webseiten könnten externen Code in verwundbaren Addons ausführen. \textbf{Hinweis: Webseiten können sehen, welche Addons verwendet werden - sogar die, die eigentlich deaktiviert sind.}}
		\end{description}
	\end{frame}

\begin{frame}{Schwachstellen im Browser II}
	Potenzielle Angriffsflächen:
	\begin{description}
		\item[Plugins] {Plugins sind ähnlich wie Browser, werden jedoch häufig von anderer Software auf dem Computer installiert. Sie dienen zur Darstellung bestimmter Inhalte auf der Webseite. Das bekannteste und für seine vielen Schwachstellen berüchtigte Plugin ist der Flash-Player von Adobe.
		
		Noch ein Tip: Webseiten können auch alle Plugins sehen, die verwendet werden und darauf entsprechend reagieren.}
	\end{description}
\end{frame}
	\subsection{Datenschutz}
	\begin{frame} 
		\begin{itemize}
			\item{\enquote{Privacy protects us from abuses by those in power, even if we’re doing nothing wrong at the time of surveillance.}
				
			-- \emph{Bruce Schneier}}
			\item Welche Daten werden für welchen Verwendungszweck wo erhoben, wie verarbeitet und an wen weitergegeben?
			\item Tradeoff: Privatsphäre vs. Useability
		\end{itemize}
	\end{frame}
	
	\begin{frame}{Spielregeln}
		\begin{itemize}
			\item personenbezogenen Daten
			\item Verwendung
			\item notwendigen und freiwilligen Angaben
			\item Aufklärung über etwaige Nachteile
			\item Einwilligung
		\end{itemize}
	\end{frame}
	
	\begin{frame}{Rechte}
		\begin{itemize}
			\item Informationelle Selbstbestimmung
			\item Recht auf Auskunft
			\item Recht auf Berichtigung
			\item Recht auf Löschung
		\end{itemize}
	\end{frame}
	
	\begin{frame}{Praxis: Internet}
		\begin{itemize}
			\item viele Länder, viele Gesetze
			\item Welche Daten personenbezogen sind
			\item viele Spuren beim Surfen ermöglichen zuordnung
			\note{Welche Techniken des Trackings werden beim genutzt?}
			\note{Welche Maßnahmen kann man zum Schutz meiner Privatsphäre ergreifen?}
		\end{itemize}
	\end{frame}
	
	\subsection{Tracking}
	\begin{frame}{Tracking beim Verbindungsaufbau I}
		\begin{itemize}
			\item DNS-Anfragen
			\item IP-Adressen Routing
			\item Laden von externen Seiteninhalten
			\begin{itemize}
				\item Web-beacons
				\item Social-media-like-buttons
			\end{itemize}
		\end{itemize}
	\end{frame}

	\begin{frame}{Tracking beim Verbindungsaufbau II}
		\begin{figure}[H]
			\includegraphics[width=\textwidth]{Abbildungen/trackography.png}
			\label{fig:trackography}
			\caption{Verbindungen im Allgemeinen}
		\end{figure}
	\end{frame}

	\begin{frame}{Tracking beim Verbindungsaufbau III}
		\begin{figure}[H]
			\includegraphics[width=\textwidth]{Abbildungen/webbeacons.png}
			\label{fig:webbeacons}
			\caption{Social-media-like-buttons}
		\end{figure}
	\end{frame}

	
	\begin{frame}{Tracking durch verräterischen Browser I}
		\begin{itemize}
			\item Geodaten
			\item URLs 
			\item Cookies in verschiedenen Geschmacksrichtungen
			\item Browserfingerprinting: Betriebssystem, Browsersoftware, Auflösung, JavaScript / Flash, Plugins, Addons, installierte Schriften, Sprache, Cookies
		\end{itemize}
	\end{frame}
	
	\begin{frame}{Tracking durch verräterischen Browser II}
		\begin{figure}[V]
			\includegraphics[width=0.8\textwidth]{Abbildungen/url_shortener.png}
			\label{fig:url shortener}
		\end{figure}
	\end{frame}

	\begin{frame}{Tracking durch verräterischen Browser III}
		\begin{figure}[V]
			\includegraphics[scale=0.5]{Abbildungen/lightbeam.png}
			\label{fig:lightbeam}
		\end{figure}
	\end{frame}

	\begin{frame}{Tracking durch verräterischen Browser IV}
		\begin{figure}[V]
			\includegraphics[scale=0.25]{Abbildungen/panopticlick.png}
			\label{fig:panopticlick}
		\end{figure}
	\end{frame}
	
	\subsubsection{\enquote{Angewandter Datenschutz} im Internet}
	
		\begin{frame}{Todo-Platzhalter}
			\begin{itemize}
				\item sicherheitsrelevante Ratschläge einbauen

			\end{itemize}
		\end{frame}
	
	\section{Counterstrike}
	\begin{frame}
		\centering \huge \textbf{Counterstrike}
	\end{frame}
	\subsection{Verhalten im Netz}
	\begin{frame}{Verhalten im Netz}
		\begin{itemize}
			\item Datensparsamkeit
			\begin{itemize}
				\item weniger ist mehr
				\item nicht überall registrieren
				\item nicht eingeloggt bleiben
			\end{itemize}
			\item Kostenlose Dienste hinterfragen
			\begin{itemize}
				\item Suchmaschinen: duckduckgo.com
				\item Email: mailbox.org / poesto.de
				\item Cloud: wenn dann verschlüsselt
			\end{itemize}
		\end{itemize}
	\end{frame}
	
	\subsection{Technische Hilfsmittel}
	\begin{frame}{Addons Firefox}
		\noindent
		\begin{description}
			\item[PrivaConf] \url{https://addons.mozilla.org/de/firefox/addon/privaconf/}
			\item[Cookie-Controller] \url{https://addons.mozilla.org/de/firefox/addon/cookie-controller/}
			\item[Canvasblocker] \url{https://addons.mozilla.org/de/firefox/addon/canvasblocker/}
			\item[no-resource-uri-leak] \url{https://addons.mozilla.org/de/firefox/addon/no-resource-uri-leak/}
			\item[UserAgentChanger]{... Finger weg!}
		\end{description}
	\end{frame}
	
	\subsection{Links für Interessierte}
	\begin{frame}{Datenkraken}
			\noindent
			\begin{itemize}
				\item Google: \href{https://myactivity.google.com/}{Alles, was du bei Google gemacht hast}
				\item facebook: \href{https://netzpolitik.org/2016/98-daten-die-facebook-ueber-dich-weiss-und-nutzt-um-werbung-auf-dich-zuzuschneiden/}{Was die so für Daten sammeln}
				\item Apple: \href{https://www.apple.com/privacy/privacy-policy/}{Datenschutzbestimmungen}
			\end{itemize}
	\end{frame}
	
		\begin{frame}{much knowledge -- so interactive -- wow}
			\noindent
			\begin{itemize}
				\item Browserfingerprinting \href{https://panopticlick.eff.org}{panopticlick.eff.org}
				\item \href{https://sec.hpi.de/vulndb/sd_first/}{Browser-Sicherheits-Check}
				\item Nachrichtenlesen: \href{https://trackography.org/}{Visualisierung von Trackern}
				\item \href{https://www.mbem.nrw/unterwegs-im-oeffentlichen-wlan-aber-gut-geschuetzt}{Ratgeber fürs Surfen im öffentlichem WLAN }
				\item Alles, was wir wissen, wissen wir vom \href{https://privacy-handbuch.de/}{privacy Handbuch}
			\end{itemize}
		\end{frame}
	
	% Da dies ein Vorlage für beliebige Vorträge ist, lassen sich kaum
	% allgemeine Regeln zur Strukturierung angeben. Da die Vorlage für
	% einen Vortrag zwischen 15 und 45 Minuten gedacht ist, fährt man aber
	% mit folgenden Regeln oft gut.  
	
	% - Es sollte genau zwei oder drei Abschnitte geben (neben der
	%   Zusammenfassung). 
	% - *Höchstens* drei Unterabschnitte pro Abschnitt.
	% - Pro Rahmen sollte man zwischen 30s und 2min reden. Es sollte also
	%   15 bis 30 Rahmen geben.
	
	
\end{document}


