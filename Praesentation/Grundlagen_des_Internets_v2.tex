% Todo:
% - ganz am Anfang ggf. Logos einbetten: https://download.terminal21.de/terminal21/logo/logo_aktuell_schwarz.svg
% - anschauliche Linkliste
% - ganz am Ende wäre ggf. ein Quellen-Nachweis aller Bilder schön

%hierbei handelt es sich um die Präsentation imt .tex-Format
%Abbildungen etc. müssen entsprechend in eigenen Unterordnern präsent sein!

\documentclass[hyperref={colorlinks,linkcolor=white}, utf8]{beamer}

\mode<presentation>
{
	\usetheme{Warsaw}
	% oder ...
	
	\setbeamercovered{transparent}
	% oder auch nicht
}

\usepackage[ngerman]{babel}
\usepackage[utf8]{inputenc}
\usepackage[T1]{fontenc}
\usepackage{times}
\usepackage[babel, german=quotes]{csquotes}


\title{Grundlagen des Internets und Sicherheit im Web}
\subtitle{Das Böse im Netz und was man dagegen tun kann} % (optional)
\author{Georg A. Murzik, Marcus Schilling}
\institute{Terminal.21}

\date{05. November 2016}

% Dies wird lediglich in den PDF Informationskatalog einfügt. Kann gut
% weggelassen werden.
\subject{Grundlagen des Internets v2}

%Logo vom Terminal.21
\pgfdeclareimage[height=1.0cm]{Logo_Terminal}{Abbildungen/Logo_Terminal}
\logo{\pgfuseimage{Logo_Terminal}}

% Folgendes sollte gelöscht werden, wenn man nicht am Anfang jedes
% Unterabschnitts die Gliederung nochmal sehen möchte.
%\AtBeginSubsection[]
%{
%  \begin{frame}<beamer>{Gliederung}
%    \tableofcontents[currentsection,currentsubsection]
%  \end{frame}
%}

%Foliennummern
\setbeamertemplate{footline}
{%
	\begin{beamercolorbox}[wd=0.5\textwidth,ht=3ex,dp=1.5ex,leftskip=.5em,rightskip=.5em]{author in head/foot}%
		\usebeamerfont{author in head/foot}%
		\insertframenumber{} von \inserttotalframenumber\hfill\insertshortauthor%
	\end{beamercolorbox}%
	\vspace*{-4.5ex}\hspace*{0.5\textwidth}%
	\begin{beamercolorbox}[wd=0.5\textwidth,ht=3ex,dp=1.5ex,left,leftskip=.5em]{title in head/foot}%
		\usebeamerfont{title in head/foot}%
		\insertshorttitle%
	\end{beamercolorbox}%
}

%%%%%%%%%%%%%%%%%%%%%%%%%%%%%%%%%%%%%%%%%%%%%%%%%%%%%%%%%%%%%%%%%%%%%%%%%%%%%%%%%%%%%%%%%%%%%%%
%%%                                Begin der Präsentation                                        %%%
%%%%%%%%%%%%%%%%%%%%%%%%%%%%%%%%%%%%%%%%%%%%%%%%%%%%%%%%%%%%%%%%%%%%%%%%%%%%%%%%%%%%%%%%%%%%%%%

\begin{document}
	
	\begin{frame}
		\titlepage
	\end{frame}
	
	\section{Grundlagen des Internets}
	\begin{frame}
		\centering \huge \textbf{Grundlagen des Internets}
	\end{frame}
	
	\subsection{Aufbau des Internets}
	\begin{frame}{Das wichtigste zuerst}
		\begin{itemize}
			\item Das Internet ist keine Wolke!
			\item Alles wird von Computern gesteuert.
			\item Jeder Computer gehört jemandem.
			\item Jeder verfolgt eigene Interessen.
			\item Es sind stets wesentlich mehr Personen beteiligt, als es scheint.				\item Daten, die sichtbar sind, werden gelesen.	
		\end{itemize}		
	\end{frame}
	
	\begin{frame}{Woraus besteht das Web?}
		\begin{enumerate}
			\item Webdienste
			\item DNS
			\item Provider
			\item Clients / Hosts
		\end{enumerate}
	\end{frame}
		
	\begin{frame}{Webdienste}
		\begin{itemize}
			\item Bekannteste Dienste sind Webseiten
			\item Cloudspeicher, Handyapps, Navigation, Streaminganbieter sind ebenfalls Webdienste
			\item Webdienste sind über IP-Adressen zu erreichen
			\item IP-Adressen kann sich aber keiner merken, deswegen werden normalerweise URLs verwendet.
		\end{itemize}		
	\end{frame}
	
	\begin{frame}{Webclients}
		\begin{itemize}
			\item Schnittstelle zwischen Nutzer und Internet
			\item Bekannteste Clients sind Webbrowser
			\item Clients stellen Daten von Webservern übersichtlich dar und bieten Interaktionsmöglichkeiten mit dem Server
		\end{itemize}
	\end{frame}
	
	\begin{frame}{DNS}
		\begin{itemize}
			\item DNS (Domain Name Service) dient als eine Art Telefonbuch des Internets
			\item Wandelt Domains in URLs in IP-Adressen um
			\item Jeder Computer, der mit dem Internet kommunizieren möchte, muss mindestens einen DNS kennen.
		\end{itemize}
	
		\begin{figure}[H]
			\includegraphics[width=0.35\textwidth]{Abbildungen/Browser_DNS_Server_Communication}
		\end{figure}				
	\end{frame}
	
	\begin{frame}{Wer hängt zwischen mir und der Webseite?}
		\begin{figure}[H]
			\includegraphics[width=\textwidth]{Abbildungen/Uebersicht_Internet}
			\label{fig:Übersicht des Internets}
		\end{figure}
	\end{frame}
	
	\section{Gefahren im Netz}
	\begin{frame}
		\centering \huge \textbf{Gefahren im Netz}
	\end{frame}
	
	\begin{frame}{Mangelnde Sicherheitsstandards - Infrastruktur}
		\alert{Die Protokolle und die Infrastruktur des Internets sind sehr alt!}
		
		\begin{itemize}
			\item Ursprüngliche Protokolle sahen kaum Bedarf für Internetsicherheit
			\item Daher keine Authentifizierungs- oder Verschlüsselungsmechanismen
			\item Heutige Sicherheitstechniken wurden nachträglich implementiert und sitzen eher \enquote{oben auf}.
		\end{itemize}
	
		\begin{block}{Entwicklung des WWW}
			Das Web wurde ursprünglich für Forscher entwickelt, die ihre Studienergebnisse untereinander austauschen wollten. Erst durch die allmähliche Verbreitung über den Globus erkannte man, dass gewisse Sicherheitsstandards eingehalten werden mussten.
		\end{block}		
	\end{frame}

	\begin{frame}{Mangelnde Sicherheitsstandards - Software}
		Nicht nur die Infrastruktur weißt Lücken auf. Selbst aktuellste Software wird teilweise stümperhaft programmiert.
		\begin{itemize}
			\item Schlechte Programmierung ist kaum mit mangelnden Standards zu erklären.
			\item Software soll immer schneller mit weniger Aufwand produziert werden.
			\item Software wird immer komplexer.
			\item Politik hat sich gewandelt. Software ist nun eher ein Service als ein Produkt.
		\end{itemize}
	\end{frame}

	\begin{frame}{Übersicht der Gefahren im Netz}
		\begin{block}{Was wollen Angreifer?}
			Unabhängig von der Art der Schwachstelle ist die Hauptgefahr fast immer dieselbe: Dass Daten in die falschen Hände geraten. Solche Daten könnten z.B. Bank-, Login-, oder Adressdaten sowie allgemeine persönliche Daten sein. Ein anderes Ziel ist die Nutzung der Infrastruktur bspw. für Botnetze.
		\end{block}
		
		Es gibt viele Wege, wie man als Angreifer an solche Informationen gelangen kann:
		\begin{enumerate}
			\item Phishing
			\item Ausnutzen von Schwachstellen
			\item Tracking			
		\end{enumerate}
	\end{frame}

	\subsection{Phishing}
	\begin{frame}
		\begin{block}{Was ist Phishing?}
			Phishing ist der Versuch, Daten zu stehlen, indem sich ein Angreifer das Vertrauen des Opfers erschleicht, indem er sich für eine vertrauenswürdige Kontaktperson ausgibt.
		\end{block}
		
		\begin{itemize}
			\item gefälschte E-Mails
			\item gefälschte Webseiten
			\item gefälschte Nachrichten
		\end{itemize}
	
		\alert{Ziele sind die eben genannten Daten der Opfer.}		
	\end{frame}

	\begin{frame}{Live-Demo}
		\begin{itemize}
			\item Phishing-Angriffe sind recht einfach
			\item Einfache Angriffe funktionieren vor allem bei Massenangriffen gut
			\item Gezielte Attacken setzen viel Vorwissen über das Opfer voraus
		\end{itemize}
	
		\begin{exampleblock}{Beispiel: Studentenfischen}
			Ich zeige nun, wie wir uns Zugriff auf Studentenlogins verschaffen könnten.
		\end{exampleblock}
	\end{frame}
	
	\begin{frame}{Abwehr von Phishing-Attacken}
		\begin{alertblock}{Entwarnung(?)}
			Phishing-Seiten unterscheiden sich von realen Seiten!
		\end{alertblock}
	
		Vor allem wegen der kurzen Zeit unterscheidet sich die falsche Seite von der realen in folgenden Punkten:
		\begin{enumerate}
			\item URL
			\item Verschlüsselung / Zertifikat
			\item Verhalten
		\end{enumerate}		
	\end{frame}
	
	\begin{frame}{Abwehr von Phishing-Attacken II}
		\alert{Was man tun kann:}
		\begin{enumerate}
			\item Misstrauisch sein \& Nachdenken!
			\item Links mit dem Mouse-Hover-Test prüfen, bevor man sie anklickt.
			\item URL prüfen.
			\item Im Falle von E-Mails den vermeintlichen Sender selbst noch einmal kontaktieren (nicht über angebotene Links!).
		\end{enumerate}
	\end{frame}
	
	\subsection{Ausnutzen von Schwachstellen}
	\begin{frame}
		Deutlich schwieriger abzuwehren sind solche Angriffe, die \emph{Schwachstellen} der Internetseite selbst, des Browsers oder sonstige Programme des Computers ausnutzen.
	\end{frame}

	\begin{frame}{Was sind \enquote{Schwachstellen}!?}
		Schwachstellen sind nichts anderes als Programmierfehler, genauer gesagt, Anwendungsfälle, die der oder die Programmierer nicht bedacht haben. 
		
		Bekommen Programme andere Eingaben, als erwartet und können sie damit nicht korrekt umgehen, so stürzen sie meist ab oder versuchen, die Daten so einzulesen, wie sie es eigentlich nicht tun sollten. Handelt es sich bei einem erwarteten Nutzernamen auf einer Webseite plötzlich um Programmcode, so sollte der Server der Webseite diesen natürlich nicht ausführen. Viele machen das aber (Stichwort: \emph{SQL Injection}).		
	\end{frame}

	\begin{frame}{Was kann passieren?}
		Werden solche Lücken ausgenutzt, so bedeutet das, dass spezieller Code bspw. in präparierten Webseiten direkt durch den Browser oder gar das Betriebssystem selbst ausgeführt wird und bspw. einen Trojaner installiert. Gelangt dieser auf den Rechner des Nutzers, entstehen erheblich größere Gefahren (Ausspähen der Nutzeraktivität und Dateien auf dem Computer usw.).
	\end{frame}

	\begin{frame}{Einschub: Trojaner vs. Viren}
		Viren sind darauf angewiesen, vom Nutzer selbst heruntergeladen zu werden. Sie sind versteckt in vertrauenserweckenden Dateien verschiedener Tauschbörsen wie Videos, Bilder oder besonders kostengünstiger Spiele alternativer Downloadplattformen.
		
		Sie können aber auch in Anhängen von Mails wie PDF- oder MS Office-Dateien lauern und sich dann beim Öffnen nebenher installieren, wenn sie denn eine Schwachstelle in Office oder dem PDF-Viewer ausnutzen können.
	\end{frame}

	\begin{frame}{Einschub: Trojaner vs. Viren II}
		Trojaner sind bösartige Programme, die sich in anderen Programmen verstecken und böses auf dem Computer anrichten. Sie können sich mit dem Internet verbinden, mit ihrem Erschaffer kommunizieren und Befehle ausführen oder einfach fleißig Daten vom Rechner klauen.
		
		Sie sind gerne versteckt in diversen Crackprogrammen und Keygeneratoren, nach denen viele sogar gezielt suchen.
	\end{frame}

	\begin{frame}{Schwachstellen im Browser}
		Browser setzen vorrangig auf Kommunikation mit dem Internet, das Öffnen von Dateien aus dem Internet und dem Ausführen kleiner Programme (Javascript) aus dem Internet.
		
		Klingt nach der vollen Drönung.
		
		Gerade Browser besitzen daher viele Angriffsflächen für Schadcode. 
		\begin{description}
			\item[AddOns] {Addons sind Erweiterungen des Browsers, die Zugriff auf die Inhalte der Webseite bekommen, bevor der Nutzer sie zu sehen bekommt. Ihre Fähigkeiten reichen von der Integration zusätzlicher Emojis über das Blockieren von Werbung bis hin zum Umleiten der gesamten Kommunikation über Proxies. 
			
			Präparierte Webseiten könnten externen Code in verwundbaren Addons ausführen. \textbf{Hinweis: Webseiten können sehen, welche Addons verwendet werden - sogar die, die eigentlich deaktiviert sind.}}
		\end{description}
	\end{frame}

\begin{frame}{Schwachstellen im Browser II}
	Potenzielle Angriffsflächen:
	\begin{description}
		\item[Plugins] {Plugins sind ähnlich wie Browser, werden jedoch häufig von anderer Software auf dem Computer installiert. Sie dienen zur Darstellung bestimmter Inhalte auf der Webseite. Das bekannteste und für seine vielen Schwachstellen berüchtigte Plugin ist der Flash-Player von Adobe.
		
		Noch ein Tip: Webseiten können auch alle Plugins sehen, die verwendet werden und darauf entsprechend reagieren.}
	\end{description}
\end{frame}
	
	\subsection{Datenschutz}
	\begin{frame} 
		\begin{itemize}
			\item{\enquote{Privacy protects us from abuses by those in power, even if we’re doing nothing wrong at the time of surveillance.}
				
				-- \emph{Bruce Schneier}}
			\item Welche Daten werden für welchen Verwendungszweck wo erhoben, wie verarbeitet und an wen weitergegeben?
			
			
			
			\item Tradeoff: Privatsphäre vs. Useability
		\end{itemize}
	\end{frame}
	
	\begin{frame}{Spielregeln}
		\begin{itemize}
			
			
			\item personenbezogenen Daten
			\item Verwendung
			\item notwendigen und freiwilligen Angaben
			\item Aufklärung über etwaige Nachteile
			\item Einwilligung
			
		\end{itemize}
	\end{frame}
	
	\begin{frame}{Rechte}
		\begin{itemize}
			\item Informationelle Selbstbestimmung
			
			\item Recht auf Auskunft
			\item Recht auf Berichtigung
			
			\item Recht auf Löschung
		\end{itemize}
	\end{frame}
	
	\begin{frame}{Praxis: Internet}
		\begin{itemize}
			\item viele Länder, viele Gesetze
			
			
			\item Welche Daten personenbezogen sind
			\item viele Spuren beim Surfen ermöglichen zuordnung
			\note{Welche Techniken des Trackings werden beim genutzt?}
			\note{Welche Maßnahmen kann man zum Schutz meiner Privatsphäre ergreifen?}
		\end{itemize}
	\end{frame}
	
	\subsection{Tracking}
	\begin{frame}{Tracking beim Verbindungsaufbau I}
		\begin{itemize}
			\item DNS-Anfragen
			\item IP-Adressen Routing
			\item Laden von externen Seiteninhalten
			\begin{itemize}
				\item Web-beacons
				\item Social-media-like-buttons
			\end{itemize}
		\end{itemize}
	\end{frame}
	
	\begin{frame}{Tracking beim Verbindungsaufbau II}
		\begin{figure}[H]
			\includegraphics[width=\textwidth]{Abbildungen/trackography.png}
			\label{fig:trackography}
			\caption{Verbindungen im Allgemeinen}
		\end{figure}
	\end{frame}
	
	\begin{frame}{Tracking beim Verbindungsaufbau III}
		\begin{figure}[H]
			\includegraphics[width=\textwidth]{Abbildungen/webbeacons.png}
			\label{fig:webbeacons}
			\caption{Social-media-like-buttons}
		\end{figure}
	\end{frame}
	
	
	\begin{frame}{Tracking durch verräterischen Browser I}
		\begin{itemize}
			\item Geodaten
			\item URLs 
			\item Cookies in verschiedenen Geschmacksrichtungen
			\item Browserfingerprinting: Betriebssystem, Browsersoftware, Auflösung, JavaScript / Flash, Plugins, Addons, installierte Schriften, Sprache, Cookies
		\end{itemize}
	\end{frame}
	
	\begin{frame}{Tracking durch verräterischen Browser II}
		\begin{figure}[V]
			\includegraphics[width=0.8\textwidth]{Abbildungen/url_shortener.png}
			\label{fig:url shortener}
		\end{figure}
	\end{frame}
	
	\begin{frame}{Tracking durch verräterischen Browser III}
		\begin{figure}[V]
			\includegraphics[scale=0.5]{Abbildungen/lightbeam.png}
			\label{fig:lightbeam}
		\end{figure}
	\end{frame}
	
	\begin{frame}{Tracking durch verräterischen Browser IV}
		\begin{figure}[V]
			\includegraphics[scale=0.25]{Abbildungen/panopticlick.png}
			\label{fig:panopticlick}
		\end{figure}
	\end{frame}
	
	\subsubsection{\enquote{Angewandter Datenschutz} im Internet}
	
	\section{Counterstrike}
	\begin{frame}
		\centering \huge \textbf{Counterstrike}
	\end{frame}
	
	
	\subsection{Verhalten im Netz und technische Hilfsmittel}
		\begin{frame}{Sicherheitsmaßnahmen}
			\begin{itemize}
				\item Software aktuell halten
				\item URLs vor dem Klicken überprüfen
				\item JavaScript u. Flash im Ausnahmefall erlauben
				\item Nachladen von Schriften deaktivieren
				\item verschiede Passwörter nutzen
				\item keine komischen Sachen runterladen und ausführen
			\end{itemize}
		\end{frame}
	
	\begin{frame}{Datenschutzmaßnahmen}
		\begin{itemize}
			\item Datensparsamkeit
			\begin{itemize}
				\item weniger ist mehr
				\item nicht überall registrieren
				\item nicht eingeloggt bleiben
			\end{itemize}
			\item Kostenlose Dienste hinterfragen
			\begin{itemize}
				\item Suchmaschinen: \url{https://duckduckgo.com}
				\item Email: \url{https://mailbox.org}, \url{https://poesto.de}
				\item Cloud: verschlüsselt und oder selbst betreiben
			\end{itemize}
			\item auf HTTPS achten
			\item Cookies verbieten, Ausnahmen erlauben
			\item Verlauf und Cache sitzungsweise löschen lassen
			\item JavaScript u. Flash deaktivieren
			\item Plugins deaktivieren
		\end{itemize}
	\end{frame}
	
	\begin{frame}{Addons Firefox}
		\noindent
		\begin{description}
			\item[PrivaConf] \url{https://addons.mozilla.org/de/firefox/addon/privaconf/}
			\item[Cookie-Controller] \url{https://addons.mozilla.org/de/firefox/addon/cookie-controller/}
			\item[Canvasblocker] \url{https://addons.mozilla.org/de/firefox/addon/canvasblocker/}
			\item[no-resource-uri-leak] \url{https://addons.mozilla.org/de/firefox/addon/no-resource-uri-leak/}
			\item[No-Script] \url{https://addons.mozilla.org/en-US/firefox/addon/noscript/}
			\item[UserAgentChanger]{... Finger weg!}
		\end{description}
	\end{frame}
	
	\subsection{Links für Interessierte}
	\begin{frame}{Datenkraken}
			\noindent
			\begin{itemize}
				\item Google: \href{https://myactivity.google.com/}{Alles, was du bei Google gemacht hast}
				\item facebook: \href{https://netzpolitik.org/2016/98-daten-die-facebook-ueber-dich-weiss-und-nutzt-um-werbung-auf-dich-zuzuschneiden/}{Was die so für Daten sammeln}
				\item Apple: \href{https://www.apple.com/privacy/privacy-policy/}{Datenschutzbestimmungen}
			\end{itemize}
	\end{frame}
	
		\begin{frame}{much knowledge -- so interactive -- wow}
			\noindent
			\begin{itemize}
				\item Browserfingerprinting \href{https://panopticlick.eff.org}{panopticlick.eff.org}
				\item Hasso-Plattner-Institut: \href{https://sec.hpi.de/vulndb/sd_first/}{Browser-Sicherheits-Check}
				\item Online-Nachrichten: \href{https://trackography.org/}{Visualisierung von Trackern}
				\item Ratgeber: \href{https://www.mbem.nrw/unterwegs-im-oeffentlichen-wlan-aber-gut-geschuetzt}{Surfen im öffentlichem WLAN }
				\item Nachschlagewerk zum Tracking: \href{https://privacy-handbuch.de/}{privacy Handbuch}
				\item \url{https://download.terminal21.de/workshops/}
			\end{itemize}
		\end{frame}
	
	% Da dies ein Vorlage für beliebige Vorträge ist, lassen sich kaum
	% allgemeine Regeln zur Strukturierung angeben. Da die Vorlage für
	% einen Vortrag zwischen 15 und 45 Minuten gedacht ist, fährt man aber
	% mit folgenden Regeln oft gut.  
	
	% - Es sollte genau zwei oder drei Abschnitte geben (neben der
	%   Zusammenfassung). 
	% - *Höchstens* drei Unterabschnitte pro Abschnitt.
	% - Pro Rahmen sollte man zwischen 30s und 2min reden. Es sollte also
	%   15 bis 30 Rahmen geben.
	
\end{document}


